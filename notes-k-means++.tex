\documentclass[9pt]{article}

\usepackage{fullpage,amssymb,amsmath,amsthm}
\usepackage{algorithm,algorithmic}

\newtheorem{lemma}{Lemma}
\newtheorem{theorem}[lemma]{Theorem}

\newcommand{\R}{\mathbb{R}}
\newcommand{\norm}[1]{\|{#1}\|}

\DeclareMathOperator*{\argmin}{argmin}
\DeclareMathOperator*{\Exp}{\mathbf{E}}

\title{Notes on $k$-means++ algorithm}
\author{D\'avid P\'al}

\begin{document}

\maketitle

\section{$K$-means problem and Lloyd's heuristic}

Given two finite sets $S,C \subset \R^d$ we define the \emph{cost of $C$ on
$S$} as
$$
f(S,C) = \sum_{x \in S} \min_{c \in C} \norm{x-c}^2 \; .
$$
Here and in the rest of the note, $\norm{\cdot}$ denotes the Euclidean norm in
$\R^d$; that is $\norm{v} = \sqrt{\sum_{i=1}^d v_i^2}$.  The elements of $S$
are called \emph{input points}. The elements of $C$ are called \emph{cluster
centers}.

\textbf{The $k$-means problem} is the following optimization problem: \emph{Given a
finite set $S \subset \R^d$ and a positive integer $k$, find a set $C \subset
\R^d$ such that $|C| \le k$ that minimizes $f(S,C)$.}

The $k$-means problem is solved in practice using a local-search heuristic,
called Lloyd's algorithm (which is sometimes referred to as \emph{the $k$-means
algorithm}).

\begin{algorithm}[h]
\caption{Lloyd's algorithm a.k.a. the $k$-means algorithm
\label{algorithm:lloyd}}
\begin{algorithmic}[1]
{
\REQUIRE Finite set $S \subset \R^d$ and a positive integer $k$.
\STATE{Initialize $c_1, c_2, \dots, c_k \in \R^d$ somehow}
\WHILE{true}
\STATE{For each $i=1,2,\dots,k$, let $C_i$ be the set of points $x \in S$ that are closer
to $c_i$ that to any other $c_j$ for all $j \neq i$.}
\STATE{For each $i=1,2\dots,k$, set $c_i = \frac{1}{|C_i|} \sum_{x \in C_i} x$. If none of the centers has changed, stop.}
\ENDWHILE
\STATE{Output $c_1, c_2, \dots, c_k$}
}
\end{algorithmic}
\end{algorithm}

In step $3$ of the algorithm, if a point $x \in S$ is closest to two (or more)
different centers, the algorithm breaks ties arbitrarily according to
consistent tie breaking rule (e.g. using cluster with smaller index $i$).

The algorithm does alternating minimization.  This can be seen by defining a cost
function
$$
g(C_1, C_2, \dots, C_k; c_1, c_2, \dots, c_k) = \sum_{i=1}^k \sum_{x \in C_k} \norm{x - c_i}^2 \; .
$$
Note that $g(C_1, C_2, \dots, C_k; c_1, c_2, \dots, c_k) \ge f(\bigcup_{i=1}^k
C_i, \{c_1, c_2, \dots, c_k\})$ with equality when each $C_i$ consists of
points that are closer (or equally close) to $c_i$ that any other $c_j$. In
steps $3$ and $4$ the cost $g(\cdot;\cdot)$ decreases by minimization over the
first or the second argument separately.  When none of the two possible
minimization would decrease the function value, the algorithm stops.
Specifically, in steps $3$, the algorithm holds $c_1, c_2, \dots, c_k$ fixed
and it finds $C_1, C_2, \dots, C_k$ such that $g(C_1, C_2, \dots, C_k; c_1,
c_2, \dots, c_k)$ is minimized (subject to the constraint that $C_1, C_2,
\dots, C_k$ is a partition of $S$). This is clearly achieved by assigning each
point $x \in \bigcup_{i=1}^k C_i$ to center $c_i$ closest to it. Similarly, in
step $4$, the algorithm holds $C_1, C_2, \dots, C_k$ fixed and finds $c_1, c_2,
\dots, c_k$ that minimizes $g(C_1, C_2, \dots, C_k; c_1, c_2, \dots, c_k)$.  As
the next lemma shows the minimum is attained for $c_i = \frac{1}{|C_i|} \sum_{x
\in C_i} x$.

\begin{lemma}[Center of Mass]
\label{lemma:center-of-mass}
Let $A \subset \R^d$ be a finite set of points. Let $\mu = \frac{1}{|A|}
\sum_{x \in A} x$ be the center of mass of $A$. Then for any $c \in \R^d$,
$$
\sum_{x \in A} \norm{x - c}^2 = |A| \cdot \norm{c - \mu}^2 + \sum_{x \in A} \norm{x - \mu}^2 \; .
$$
In particular,
$$
\argmin_{c \in \R^d} \sum_{x \in A} \norm{x - c}^2 = \mu \; .
$$
\end{lemma}

\begin{proof}
We have
\begin{align*}
\sum_{x \in A} \norm{x - c}^2
& = \sum_{x \in A} \norm{x - \mu + \mu - c}^2 \\
& = \sum_{x \in A} \left( \norm{x - \mu}^2  + \norm{\mu - c}^2 + 2 \langle x - \mu, \mu - c \rangle \right) \\
& = |A| \cdot \norm{c - \mu}^2 +  \sum_{x \in A} \norm{x - \mu}^2  + 2 \sum_{x \in A} \langle x - \mu, \mu - c \rangle \\
& = |A| \cdot \norm{c - \mu}^2 +  \sum_{x \in A} \norm{x - \mu}^2  + 2 \left\langle \mu - c,  \underbrace{\sum_{x \in A} (x - \mu)}_{=0} \right\rangle \; .
\end{align*}
\end{proof}

The lemma is essentially the parallel axis theorem (also called Steiner's
theorem) from physics. In language of probability theory, it expresses the
second moment of a random variable around a point in terms of variance and
distance from of the point from the mean.

\section{$K$-means++ algorithm}

The $k$-means++ algorithm is a particular version of Lloyd's heuristic where in
step $1$ of the algorithm, the centers $c_1, c_2, \dots, c_k$ are chosen
in a particular random way.

\begin{algorithm}[h]
\caption{$k$-mean++ initialization \label{algorithm:k-means++}}
\begin{algorithmic}[1]
{
\REQUIRE Finite set $S \subset \R^d$ and a positive integer $k$.
\STATE{Choose $c_1$ uniformly at random from $S$}
\FOR{$i=2,3,\dots,k$}
\STATE{For any $x \in S$ define $D_{i-1}(x) = \min_{1 \le j \le i-1} \norm{x - c_j}$}
\STATE{Choose $c_i = c$ at random from $S$ according to the probability distribution $p(c) = \frac{(D_{i-1}(c))^2}{\sum_{x \in S} (D_{i-1}(x))^2}$}
\ENDFOR
\STATE{Output $c_1, c_2, \dots, c_k$}
}
\end{algorithmic}
\end{algorithm}

The algorithm maintains for any point $x$ its distance to the closest center.
Namely, $D_i(x)$ is the distance to the closest center in $\{c_1, c_2, \dots,
c_i\}$. Given $D_{i-1}(\cdot)$ and a new center $c_i$ we can compute $D_i(\cdot)$ in $O(d|S|)$
time, by the formula $D_i(x) = \min\{D_{i-1}(x), \norm{x - c_i}\}$.
The sampling step (step $4$) can be implemented in $O(|S|)$
time assuming we can generate a random number from the uniform distribution
over $[0,1]$ in constant time. Hence, the algorithm can be implemented in
$O(dk|S|)$ time.

The center $c_i$ is chosen according to the so-called \emph{$D^2$-weighting}:
Point $x \in S$ is chosen as a new center with probability proportional to
$(D_{i-1}(x))^2$. $D^2$-weighting is the main ingredient of $k$-means++.
Intuitively speaking, it favors a set of centers that is well spread out across
the input set $S$.

The main result about the $k$-means++ initialization is that the set of centers
it produces is an $O(\log k)$ approximate solution to the $k$-means problem. In
practice, one would run the Lloyd's heuristic until convergence initialized
with Algorithm~\ref{algorithm:k-means++}.  However, for the theoretical result
this is not necessary.

\begin{theorem}[$O(\log k)$ approximation]
\label{theorem:main}
Let $S \subset \R^d$ be a finite set of points and let $k$ be a positive integer.
Let $C^*$ be the set of $k$ centers minimizing the cost $f(S,\cdot)$.
Let $C_{++}$ be the set of centers computed with $k$-means++ initialization.
$$
f(S, C_{++}) \le (8 \ln k + 2) \cdot f(S, C^*) \; .
$$
\end{theorem}

The proof of the theorem is a bit tricky. We need a few lemmas to prove it.

We first show that choosing that if $k=1$, the algorithm's solution is (in
expectation) an $2$-approximation of the optimal solution. Since the first point
is chosen uniformly at random, this is an interesting special case.

\begin{lemma}[One random center]
\label{lemma:one-random-center}
Let $A \subset \R^d$ be a finite set of points. Let $\mu = \frac{1}{|A|} \sum_{x \in A} x$
be the center of mass of $A$. Let $c$ be a point chosen uniformly at random from $A$. Then,
expected cost of $\{c\}$ on $A$ satisfies
$$
\Exp[f(A,\{c\})] = 2 \cdot f(A, \{\mu\}) \; .
$$
\end{lemma}

\begin{proof}
We have
\begin{align*}
\Exp[f(A,\{c\})]
& = \Exp\left[\sum_{x \in A} \norm{x - c}^2\right] \\
& = \frac{1}{|A|} \sum_{c \in A} \sum_{x \in A} \norm{x - c}^2 \\
& = \frac{1}{|A|} \sum_{c \in A} \left[ |A| \cdot \norm{c - \mu}^2 + \sum_{x \in A} \norm{x - \mu}^2 \right] & \text{(by Lemma~\ref{lemma:center-of-mass})} \\
& = \sum_{c \in A} \norm{c - \mu}^2 + \sum_{x \in A} \norm{x - \mu}^2 \\
& = 2 \sum_{x \in A} \norm{x - \mu}^2 \\
& = 2 \cdot f(A, \{\mu\}) \; .
\end{align*}
\end{proof}

The next lemma characterizes what happens when we add a new center
according to $D^2$-weighting to an existing set of centers.

\begin{lemma}
Let $S \subset \R^d$ be a finite set. Let $c^*_1, c^*_2, \dots, c^*_k \in \R^d$
be an optimal $k$-means solution for $S$. Let $i \in \{1,2,\dots,k\}$ and
let $C^*_i = \{ x \in S ~:~
\argmin_{j=1,2,\dots,k} \norm{x - c_j^*} = i\}$ be the cluster corresponding to
$c^*_i$. Let $C \subset \R^d$ be an arbitrary non-empty finite set of cluster centers.
Let $D(x) = \min_{c' \in C} \norm{x - c'}$. Suppose we
choose $c$ at random from $C^*_i$ according to the probability distribution $p(c) =
\frac{(D(c))^2}{\sum_{x \in C_i^*} (D(x))^2}$. Then,
$$
\Exp[f(C_i^*, C \cup \{c\})] \le 8 \cdot f(C^*_i, \{c^*_i\}) \; .
$$
\end{lemma}

\begin{proof}
We can write the left-hand side as
\begin{align*}
\Exp[f(C_i^*, C \cup \{c\})]
& = \Exp\left[ \sum_{x \in C_i^*} \min\left\{D(x)^2, \norm{x-c}^2 \right\} \right] \\
& = \sum_{c \in C^*_i} \frac{(D(c))^2}{\sum_{x \in C_i^*} (D(x))^2} \sum_{x \in C_i^*} \min\left\{D(x)^2, \norm{x - c}^2\right\} \\
\end{align*}
We now use that $D(c) \le D(x) + \norm{x - c}$ by triangle inequality.
Consequently, $(D(c))^2 \le 2 (D(x))^2 + 2 \norm{x - c}^2$, which follows by
squaring the triangle inequality and upper bounding its right-hand side.
Averaging over $x \in C_i^*$, we get $(D(c))^2 \le \frac{2}{|C_i^*|} \sum_{x \in
C_i^*} (D(x))^2 + \frac{2}{|C_i^*|} \sum_{x \in C_i^*} \norm{x - c}^2$. Using this
inequality, we have
\begin{align*}
\Exp[f(C_i^*, C \cup \{c\})]
& = \sum_{c \in C^*_i} \frac{(D(c))^2}{\sum_{x \in C_i^*} (D(x))^2} \sum_{x \in C_i^*} \min\left\{D(x)^2, \norm{x - c}^2\right\} \\
& \le \frac{2}{|C^*_i|} \sum_{c \in C^*_i} \frac{\sum_{x \in C_i^*} (D(x))^2}{\sum_{x \in C_i^*} (D(x))^2} \sum_{x \in C_i^*} \min\left\{D(x)^2, \norm{x - c}^2\right\} \\
& \qquad + \frac{2}{|C^*_i|} \sum_{c \in C^*_i} \frac{\sum_{x \in C_i^*} \norm{x-c}^2}{\sum_{x \in C_i^*} (D(x))^2} \sum_{x \in C_i^*} \min\left\{D(x)^2, \norm{x - c}^2\right\} \\
& \le \frac{2}{|C^*_i|} \sum_{c \in C^*_i} \sum_{x \in C_i^*} \norm{x - c}^2 + \frac{2}{|C_i^*|} \sum_{c \in C^*_i} \frac{\sum_{x \in C_i^*} \norm{x-c}^2}{\sum_{x \in C_i^*} (D(x))^2} \sum_{x \in C_i^*} D(x)^2 \\
& = \frac{4}{|C^*_i|} \sum_{c \in C^*_i} \sum_{x \in C_i^*} \norm{x - c}^2 \\
& = \frac{4}{|C^*_i|} \sum_{c \in C^*_i} \left[ |C^*_i| \cdot \norm{c - c^*_i}^2 + \sum_{x \in C_i^*} \norm{x - c_i^*}^2 \right] & \text{(by Lemma~\ref{lemma:center-of-mass})} \\
& = 4 \sum_{c \in C^*_i} \norm{c - c^*_i}^2 + 4 \sum_{x \in C_i^*} \norm{x - c_i^*}^2 \\
& = 8 \cdot f(C^*_i, \{c^*_i\}) \; .
\end{align*}
\end{proof}

\begin{lemma}
\label{lemma:main}
Let $S \subset \R^d$ be a finite set of points. Let $c^*_1, c^*_2, \dots, c^*_k$
be the optimal $k$-means solution for $S$. For any $i$, let $C^*_i = \{ x \in S ~:~
\argmin_{j=1,2,\dots,k} \norm{x - c_j^*} = i\}$ be the cluster corresponding to
$c^*_i$. Let $c_1, c_2, \dots, c_\ell \in \R^d$ be an arbitrary set of cluster
centers. Let $U = \{ C_i^* ~:~ C_i^* \cap \{c_1, c_2, \dots, c_\ell\} =
\emptyset \}$ be the set of optimal clusters uncovered by $c_1, c_2, \dots, c_k$.
Let $t \in \{0, 1, \dots, |U|\}$ and suppose $t$ new centers $c_{\ell+1},
c_{\ell+2}, \dots, c_{\ell+t}$ centers are added using $D^2$-weighting. Let $S_u =
\bigcup_{C \in U} C$ be the points in uncovered clusters and let $S_c = S
\setminus S_u$. Then,
\begin{multline}
\label{equation:main-lemma}
\Exp \left[ f(S, \{c_1, c_2, \dots, c_{\ell+t}\}) \right]
\le (1 + H_t) \cdot \left[f(S_c, \{c_1, c_2, \dots, c_\ell\}) + 8 \cdot f(S_u, \{c^*_1, c^*_2, \dots, c^*_k\}) \right] \\
 + \frac{u-t}{u} \cdot f(S_u, \{c_1, c_2, \dots, c_\ell\} ) \; ,
\end{multline}
where $H_t = \sum_{i=1}^t \frac{1}{i}$ is the $t$-th Harmonic number.
\end{lemma}

Before we prove the lemma, we show that it implies Theorem~\ref{theorem:main}.
After first center $c_1$ is chosen by the algorithm, precisely one optimal cluster
is covered and $k-1$ optimal clusters are uncovered. Hence, $u=k-1$.
We apply the lemma with $\ell=1$ and $t=k-1$ and obtain
\begin{align*}
& \Exp \left[ f(S, \{c_1, c_2, \dots, c_k\}) \right] \\
& \quad \le (1 + H_{k-1}) \cdot \Exp\left[f(S_c, \{c_1\}) + 8 \cdot f(S_u, \{c^*_1, c^*_2, \dots, c^*_k\}) \right] & \text{(by Lemma~\ref{lemma:main})} \\
& \quad = (1 + H_{k-1}) \cdot \sum_{i=1}^k \Pr[S_c = C^*_i] \cdot \Exp\left[f(C^*_i, \{c_1\}) + 8 \cdot f(S \setminus C^*_i, \{c^*_1, c^*_2, \dots, c^*_k\}) ~\middle|~ S_c = C^*_i \right] \\
& \quad \le (1 + H_{k-1}) \cdot \sum_{i=1}^k \Pr[S_c = C^*_i] \cdot \Exp\left[2 \cdot f(C^*_i, \{c^*_i\}) + 8 \cdot f(S \setminus C^*_i, \{c^*_1, c^*_2, \dots, c^*_k\}) ~\middle|~ S_c = C^*_i \right] & \text{(by Lemma~\ref{lemma:center-of-mass})}  \\
& \quad \le (1 + H_{k-1}) \cdot \sum_{i=1}^k \Pr[S_c = C^*_i] \cdot 8 \cdot f(S, \{c^*_1, c^*_2, \dots, c^*_k\}) \\
& \quad = 8 (1 + H_{k-1}) \cdot f(S, \{c^*_1, c^*_2, \dots, c^*_k\}) \\
& \quad \le 8 (2 + \ln k) \cdot f(S, \{c^*_1, c^*_2, \dots, c^*_k\}) \; .
\end{align*}

\begin{proof}[Proof of Lemma~\ref{lemma:main}]
We prove the lemma by structural induction on pairs $(t,u)$ satisfying $u > 0$,
$t \ge 0$ and $t \le u$. Namely, in the inductive case, we show that if the
result holds for $(t-1, u)$ and for $(t-1, u-1)$ then it also holds for
$(t,u)$. There are two bases cases.  The first is $t=0$ and $u > 0$. The second
is $t=u=1$.

In the first base case ($t=0$ and $u > 0$), the inequality~\eqref{equation:main-lemma} becomes
$$
f(S, \{c_1, c_2, \dots, c_\ell\})
\le f(S_c, \{c_1, c_2, \dots, c_\ell\}) + 8 \cdot f(S_u, \{c^*_1, c^*_2, \dots, c^*_k\}) + f(S_u, \{c_1, c_2, \dots, c_\ell\})
$$
The right-hand side is
$$
f(S, \{c_1, c_2, \dots, c_\ell\}) + 8 \cdot f(S_u, \{c^*_1, c^*_2, \dots, c^*_k\}) \; ,
$$
which is clearly bigger than the left-hand side.

In the second base case ($t=u=1$), the inequality~\eqref{equation:main-lemma} becomes
$$
\Exp[f(S, \{c_1, c_2, \dots, c_\ell, c_{\ell+1}\})] \le 2 \cdot \left[f(S_c, \{c_1, c_2, \dots, c_\ell\}) + 8 \cdot f(S_u, \{c^*_1, c^*_2, \dots, c^*_k\} ) \right] \; .
$$
There is precisely one uncovered cluster. The algorithm chooses $c_{\ell+1}$ from the
uncovered cluster with probability $\frac{f(S_u, \{c_1, c_2, \dots, c_\ell\})}{f(S,
\{c_1, c_2, \dots, c_\ell\})}$ and from a covered cluster with probability
$\frac{f(S_c, \{c_1, c_2, \dots, c_\ell\})}{f(S, \{c_1, c_2, \dots, c_\ell\})}$.
We have
\begin{align*}
& \Exp[f(S, \{c_1, c_2, \dots, c_\ell, c_{\ell+1}\})] \\
& = \Pr[c_{\ell+1} \in S_u] \cdot \Exp\left[ f(S, \{c_1, c_2, \dots, c_{\ell+1}\}) ~\middle|~ c_{\ell+1} \in S_u \right]  + \Pr[c_{\ell+1} \not \in S_u] \cdot \Exp\left[f(S, \{c_1, c_2, \dots, c_{\ell+1}\}) ~\middle|~ c_{\ell+1} \not \in S_u \right] \\
& = \Pr[c_{\ell+1} \in S_u] \cdot \Exp\left[ f(S, \{c_1, c_2, \dots, c_{\ell+1}\}) ~\middle|~ c_{\ell+1} \in S_u \right]
+ \frac{f(S_c, \{c_1, c_2, \dots, c_\ell\})}{f(S, \{c_1, c_2, \dots, c_\ell\})} \cdot \Exp\left[f(S, \{c_1, c_2, \dots, c_{\ell+1}\}) ~\middle|~ c_{\ell+1} \not \in S_u \right] \\
& \le \Pr[c_{\ell+1} \in S_u] \cdot \Exp\left[ f(S, \{c_1, c_2, \dots, c_{\ell+1}\}) ~\middle|~ c_{\ell+1} \in S_u \right] + f(S_c, \{c_1, c_2, \dots, c_\ell\}) \\
& = \Pr[c_{\ell+1} \in S_u] \cdot \Exp\left[ f(S_c, \{c_1, c_2, \dots, c_{\ell+1}\}) + f(S_u, \{c_1, c_2, \dots, c_{\ell+1}\}) ~\middle|~ c_{\ell+1} \in S_u \right] + f(S_c, \{c_1, c_2, \dots, c_\ell\}) \\
& = \Pr[c_{\ell+1} \in S_u] \cdot \Exp\left[ f(S_u, \{c_1, c_2, \dots, c_{\ell+1}\}) ~\middle|~ c_{\ell+1} \in S_u \right] + 2 \cdot f(S_c, \{c_1, c_2, \dots, c_\ell\}) \\
& \le 8 \cdot \Pr[c_{\ell+1} \in S_u] \cdot f(S_u, \{c_1^*, c_2^*, \dots, c_k^*\}) + 2 \cdot f(S_c, \{c_1, c_2, \dots, c_\ell\}) \\
& \le 8 \cdot f(S_u, \{c_1^*, c_2^*, \dots, c_k^*\}) + 2 \cdot f(S_c, \{c_1, c_2, \dots, c_\ell\}) \\
& \le 16 \cdot f(S_u, \{c_1^*, c_2^*, \dots, c_k^*\}) + 2 \cdot f(S_c, \{c_1, c_2, \dots, c_\ell\}) \; .
\end{align*}

In the inductive case (i.e. $u \ge t \ge 2$ or $u > t = 1$), there are two options:
Either, the first new center $c_{\ell+1}$ falls into a covered cluster, or it falls
into a covered cluster. The probability that $c_{\ell+1}$ into a covered cluster is
$\frac{f(S_c, \{c_1, c_2, \dots, c_\ell\})}{f(S, \{c_1, c_2, \dots, c_{\ell}\})}$.
The probability that it falls into an uncovered cluster is
$\frac{f(S_u, \{c_1, c_2, \dots, c_\ell\})}{f(S, \{c_1, c_2, \dots, c_{\ell}\})}$.
That is,
\begin{align*}
& \Exp[f(S, \{c_1, c_2, \dots, c_{\ell+t}\})] \\
& = \Pr[c_{\ell+1} \in S_c] \cdot \Exp\left[ f(S,\{c_1, c_2, \dots, c_{\ell+t}\}) ~\middle|~ c_{\ell+1} \in S_c \right] + \Pr[c_{\ell+1} \in S_u] \cdot \Exp\left[f(S, \{c_1, c_2, \dots, c_{\ell+t}\}) ~\middle|~ c_{\ell+1} \in S_u \right] \\
\end{align*}
We apply inductive hypothesis for $(u, t-1)$ to the first term. Namely, the
center $c_{\ell+1}$ does not increase the cost and we simply ignore it.
\begin{align*}
& \Exp\left[ f(S,\{c_1, c_2, \dots, c_{\ell+t}\}) ~\middle|~ c_{\ell+1} \in S_c \right] \\
& \le \Exp\left[ f(S,\{c_1, c_2, \dots, c_{\ell}, c_{\ell+2}, c_{\ell+t}\}) \right] \\
& \le (1 + H_{t-1}) \cdot \left[f(S_c, \{c_1, c_2, \dots, c_{\ell}\}) + 8 \cdot f(S_u, \{c^*_1, c^*_2, \dots, c^*_k\}) \right] + \frac{u-t+1}{u} \cdot f(S_u, \{c_1, c_2, \dots, c_\ell\}) \; .
\end{align*}
To deal with the second term, we consider each of the uncovered clusters.
For any uncovered cluster $C \in U$, the probability that $c_{\ell+1}$ lies
in $C$ is $\frac{f(C, \{c_1, c_2, \dots, c_{\ell}\})}{f(S,\{c_1, c_2, \dots, c_\ell\})}$.
Hence, we have
\begin{align*}
& \Pr[c_{\ell+1} \in S_u] \cdot \Exp\left[f(S, \{c_1, c_2, \dots, c_{\ell+t}\}) ~\middle|~ c_{\ell+1} \in S_u \right] \\
& = \sum_{C \in U} \Pr[c_{\ell+1} \in C] \cdot \Exp\left[f(S, \{c_1, c_2, \dots, c_{\ell+t}\}) ~\middle|~ c_{\ell+1} = x \right] \\
& = \sum_{C \in U} \sum_{x \in C} \Pr[c_{\ell+1} = x] \cdot \Exp\left[f(S, \{c_1, c_2, \dots, c_{\ell+t}\}) ~\middle|~ c_{\ell+1} =  x \right] \\
& \le \sum_{C \in U} \sum_{x \in C} \Pr[c_{\ell+1} = x] \cdot \bigg( (1+H_{t-1}) \cdot [f(S_c \cup C, \{c_1, c_2, \dots, c_{\ell}, x\}) + 8 \cdot f(S_u \setminus C, \{c^*_1, c^*_2, \dots, c^*_k\})] \\
& \qquad \qquad + \frac{u-t}{u-1} \cdot f(S_u \setminus C, \{c_1, c_2, \dots, c_\ell, x \} ) \bigg) \\
& \le \sum_{C \in U} \sum_{x \in C} \Pr[c_{\ell+1} = x] \cdot \bigg( (1+H_{t-1}) \cdot [f(S_c, \{c_1, c_2, \dots, c_{\ell}\}) + f(C, \{c_1, c_2, \dots, c_\ell, x\}) \\
& \qquad  + 8 \cdot f(S_u, \{c^*_1, c^*_2, \dots, c^*_k\}) - 8 \cdot f(C, \{c^*_1, c^*_2, \dots, c^*_k\})] + \frac{u-t}{u-1} \cdot f(S_u \setminus C, \{c_1, c_2, \dots, c_\ell, x \} ) \bigg)
\end{align*}

\end{proof}

\end{document}
